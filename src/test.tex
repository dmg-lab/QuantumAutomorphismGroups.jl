
\appendix
\section{Computational results}
\begin{center}
    \small
    \begin{tabular}{ | l | l | l | l | l | l |p{2.5cm} |}
    \hline
    Name  & $\Aut_{\pB}^+$ &  $\Aut_{\rank}^+$  & $\Aut_{\pF}^+$ & $\Aut_{\pC}^+$ & $\Aut(M)$ & Comments \\ \hline
     u($r<2$,$n<3$) &  C & C & C & C & ? &All $\Aut_{*}^+$ are the same  \\ \hline
    u(2,3) &  C & C & C & C &  ? &All $\Aut_{*}^+$ are the same  \\ \hline
    u(2,4) &  NC &  C & C & C &  ? & r2n4 \\ \hline
    0***** &  NC &  C & C & C &  ? & r2n4 \\ \hline
    0****0 &  NC &  C & C & NC & ? & r2n4 \\ \hline
    00*0** & C &  C & C & C & ? & r2n4  \\ \hline
    000*** & C &  C & C & C &  ? &r2n4  \\ \hline
    0000** & C &  C & C & C & ? & r2n4 \\ \hline
    00000* & NC &  C & C & NC &  ? &r2n4 \\ \hline
    u(3,4) &  C &  ? & C & C &   ? &All $\Aut_{*}^+$ are the same. Flats and Circuits had to be done with the Letterplace approach. \\ \hline
     0********* & NC &  ? & ? & C & ? & r2n5, all $n>5$ matroid computation used the Letterplace method \\ \hline
     0****0**** & NC &  ? & ? & NC &  ? &r2n5 \\ \hline
     00*0**0*** & NC &  ? & ? & C &  ? &r2n5 \\ \hline
     000******* & NC &  ? & ? & C & ? & r2n5 \\ \hline
     000******0 & NC &  ? & ? & NC & ? & r2n5 \\ \hline
     0000**0*** & NC &  ? & ? & C & ? & r2n5 \\ \hline
     0000**0**0 & NC &  ? & ? & NC & ? & r2n5 \\ \hline
     00000*00** & NC &  ? & ? & NC & ? & r2n5 \\ \hline
     0000... & NC &  ? & ? & NC & ? & r2n5 \\ \hline
     0000... & C &  ? & ? & C & ? & r2n5 \\ \hline
     0000...& NC &  ? & ? & NC &  ? &r2n5 \\ \hline
     00000... & NC &  ? & ? & NC & ? & r2n5 \\ \hline
     00*0**... & NC &  ? & ? & C & ? & r2n6 \\ \hline
     00**0**...& C &  ? & ? & ? & ? & r3n6 \\ \hline
     00**0**...& C &  ? & ? & ? & ? & r3n7 \\ \hline
    fano &  ? & ? & ? & ? & ? & - \\ \hline
    non-fano &  ? & ? & ? & ? & ? & - \\ \hline
    \label{Tab:computational-results}
    \end{tabular}
\end{center}




\end{document} 
